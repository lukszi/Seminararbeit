Aus den in der Business Process Discovery Phase gesammelten unstrukturierten Daten müssen in einem nächsten Schritt strukturierte Prozessmodelle erstellt werden, die als Grundlage für die weitere Analyse dienen.
Zur Modellierung von Geschäftsprozessen gibt es verschiedene Ansätze, die im Folgenden vorgestellt werden.

\subsubsection{UML}
Die Unified Modeling Language (UML) ist ein Standard für grafische Modellierung die für die Softwareentwicklung und das Systemdesign entwickelt wurde und die Darstellung und Dokumentation der Struktur und des Verhaltens von Systemen ermöglicht~\cite{OMG2017}.
Im Zusammenhang mit der Prozessmodellierung können im speziellen Aktivitätsdiagramme verwendet werden, um Prozesse in Aktionen aufgeteilt darzustellen, wobei verschiedene Akteure durch Swimlanes repräsentiert werden können~\cite{List2006}.\\

Allerdings hat die UML bei der Modellierung von Geschäftsprozessen gewisse Einschränkungen.
Während sie umfassende Unterstützung für Kontrollfluss- und Datenperspektiven bietet, ist ihre Anwendbarkeit auf ressourcenbezogene oder organisatorische Aspekte begrenzt~\cite{Russell2006}.
Darüber hinaus kann UML Schwierigkeiten haben, einige natürliche Konstrukte zu erfassen, die in Geschäftsprozessen vorkommen, wie z. B. Fälle und Interaktionen mit der betrieblichen Umgebung~\cite{Russell2006}.


\subsubsection{BPMN (Business Process Model and Notation)}

Die Business Process Model and Notation (BPMN) 2.0 ist eine standardisiertes Modellierungssprache, die verwendet wird, um Geschäftsprozesse grafisch darzustellen.
BPMN 2.0 wurde vom Object Management Group (OMG) entwickelt und bietet eine einheitliche Sprache, um Prozesse und Aktivitäten in Unternehmen zu beschreiben.\\
Das BPMN 2.0-Schema besteht aus einer Vielzahl von Symbolen, die zur Darstellung von Ereignissen, Aktivitäten, Gateways, Datenobjekten und weiteren Konstrukten verwendet werden.
Diese Symbole werden in einer grafischen Notation angeordnet, um einen Prozessfluss zu erzeugen, der die verschiedenen Schritte des Prozesses und deren Beziehungen zueinander widerspiegelt.

\subsubsection{Event-driven Process Chain (EPC)}

EPC ist ein Prozessmodellierungskonzept, das auf der Idee basiert, dass Prozesse aus einer Reihe von Ereignissen und Funktionen bestehen.
EPCs bieten eine grafische Darstellung, die es erlaubt, den Prozessfluss auf eine visuelle und verständliche Art darzustellen.
Die Methode wird häufig eingesetzt, da sie es ermöglicht, die Abhängigkeiten zwischen Prozessschritten einfach abzubilden und somit die Identifikation von Engpässen und Schwachstellen zu erleichtern.

\subsubsection{Value Stream Mapping (VSM)}

VSM ist eine Methode, die in der Produktion eingesetzt wird, um den Wertstrom zu visualisieren und zu verbessern.
Es ist jedoch auch für die Identifikation und Optimierung von Geschäftsprozessen geeignet.
VSM bietet eine grafische Darstellung, die es ermöglicht, den Fluss von Material und Informationen durch den Prozess zu visualisieren.
Die Methode ist besonders effektiv bei der Identifikation von Verschwendung und der Optimierung von Engpässen.
\subsubsection{Conclusion}
% TODO