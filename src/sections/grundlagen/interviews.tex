Interviews sind eine weit verbreitete und wichtige Methode in der Business Process Identification, um Informationen über Arbeitsabläufe, Rollen, Verantwortlichkeiten, Herausforderungen und Verbesserungsmöglichkeiten direkt von den beteiligten Mitarbeitern zu sammeln. In diesem Abschnitt wird der Einsatz von Interviews in der BPI näher erläutert.

\subsubsection{Arten von Interviews}
Generell können Interviews in drei Haupttypen unterteilt werden~\cite[P.9]{Yousuf2015}:

\begin{itemize}
    \item \textbf{Strukturierte Interviews:} Dabei handelt es sich um Interviews, bei denen im Voraus festgelegte Fragen gestellt werden und die Antworten der Befragten in einer strukturierten Form erfasst werden. Strukturierte Interviews ermöglichen eine hohe Vergleichbarkeit der Ergebnisse, können jedoch weniger tiefgreifende Informationen liefern, da sie weniger Raum für offene Diskussionen und spontane Fragen bieten.
    \item \textbf{Halbstrukturierte Interviews:} Diese Art von Interviews kombiniert Elemente aus strukturierten und unstrukturierten Interviews. Sie werden mit einem Leitfaden durchgeführt, der als Orientierung dient. Der Interviewer wird aber ermutigt, zusätzliche Fragen zu stellen oder Themen, basierend auf den Antworten der Befragten weiter zu vertiefen. Halbstrukturierte Interviews ermöglichen eine größere Flexibilität und können tiefere Einblicke in die Geschäftsprozesse liefern, erfordern jedoch mehr Zeit und Fähigkeiten bei der Analyse der Daten.
    \item \textbf{Unstrukturierte Interviews:} Unstrukturierte Interviews sind offene Gespräche, die auf die individuellen Bedürfnisse und Interessen der Befragten zugeschnitten sind. Sie bieten den größten Raum für spontane Fragen und Diskussionen und können wertvolle qualitative Informationen liefern. Allerdings sind sie in der Datenauswertung anspruchsvoll und zeitintensiv und erfordern eine hohe Expertise des Interviewers.
\end{itemize}


\subsubsection{Vor- und Nachteile von Interviews}

Der Einsatz von Interviews zur Erhebung von Informationen bringt sowohl Vorteile als auch Nachteile mit sich:~\cite[Kapitel 5.2.2, 5.2.4]{Dumas2013}

\paragraph{Vorteile:}
Die Vorteile von Interviews sind vielseitig. Zum Einen ermöglichen Interviews einen hohen Detaillgrad der gewonnenen Informationen, indem sie die Möglichkeit bieten, durch die Befragung der involvierten Akteure Wissen zu erlangen, das in schriftlichen Dokumenten oder anderen Informationsquellen möglicherweise nicht vollständig erfasst wird. Dies trägt zur Ermittlung von Schwachstellen und ausbaufähigen Bereichen bei.\\

Andererseits zeichnen sich Interviews durch ihre Anpassungsfähigkeit aus. Im Verlauf eines Interviews können die Befragten auf unvorhergesehene Antworten oder Themen eingehen und bei Bedarf gezielte Fragen stellen, um ihr Verständnis der Prozesse zu vertiefen. Diese Flexibilität ermöglicht die Ermittlung von Details, die der Analyst sonst nicht berücksichtigt hätte.\\

Darüber hinaus bieten Interviews die Möglichkeit zur Kontextualisierung. Sie ermöglichen die Erhebung von Informationen aus der Perspektive der beteiligten Mitarbeiter. Durch die Berücksichtigung divergierender Sichtweisen wird ein besseres Verständnis der Prozesse im Kontext der Organisationskultur und -struktur erreicht. Dies trägt zur Identifizierung von Zusammenhängen und gegenseitigen Abhängigkeiten zwischen den Prozessen bei.\\

\paragraph{Nachteile:}
Im Zusammenhang mit der Identifizierung von Geschäftsprozessen zeigen sich die Nachteile von Interviews, die neben ihren Vorteilen berücksichtigt werden müssen.\\

Den wichtigsten Nachteil stellt dabei der Zeitaufwand dar. Interviews sind sowohl in der Vorbereitung und Durchführung als auch der Auswertung der gesammelten Informationen zeitintensiv. Dies kann zu Verzögerungen bei der Ermittlung und Optimierung von Geschäftsprozessen führen und stellt insbesondere bei der Untersuchung von Prozessen mit vielen Akteuren eine Herausforderung da.\\

Ein weiteres Risiko sind Biases in den erhaltenen Antworten. Die Antworten der Befragten können durch persönliche Vorlieben, Meinungen oder Erinnerungsverzerrungen beeinflusst werden, was die Objektivität der gesammelten Informationen beeinträchtigen kann. Diese Herausforderung erfordert eine sorgfältige Prüfung und Interpretation der Daten, um Ungenauigkeiten zu minimieren.\\

Die Komplexität der Auswertung stellt eine zusätzliche Herausforderung dar. Die Analyse von Interviewdaten kann komplex und zeitaufwändig sein, insbesondere wenn Teilnehmer sich in ihren Antworten selbst widersprechen oder zwischen verschiedenen Teilnehmern grobe Disparitäten auftauchen. Dies kann einen erheblichen Einsatz von Ressourcen erfordern, um sicherzustellen, dass die Analyse sowohl gründlich als auch genau ist.\\

Schließlich besteht bei Interviews das Problem der begrenzten Vergleichbarkeit von Informationen aus verschiedenen Interviews. Dies gilt insbesondere dann, wenn die Interviews von verschiedenen Interviewern durchgeführt werden oder wenn die Teilnehmer sich in ihrer Ausdrucksweise und Interpretation der Geschäftsprozesse unterscheiden. Die Sicherstellung der Konsistenz der Interviewtechniken und der Datenauswertung kann dazu beitragen, dieses Problem zu entschärfen.\\

Trotz dieser Nachteile bleiben Interviews insbesondere für die Identifikation von Geschäftsprozessen mit limitiertem Umfang ein wichtiges Instrument.

\subsubsection{Vorbereitung und Durchführung eines strukturierten Interviews}

Die Vorbereitung und Durchführung eines strukturierten Interviews lässt sich grob in die folgenden Schritte unterteilen.~\cite{Seidman2006, Kvale2009}

\paragraph{1. Zielsetzung und Planung}\
Im ersten Schritt des Interviewprozesseses spielen Zielsetzung und Planung eine entscheidende Rolle, um die Effektivität und Effizienz der Interviews zu gewährleisten. Dies gelingt wie folgt.
\begin{enumerate}
    \item \textit{Festlegung der Ziele:} Zunächst muss festgelegt werden, zu welchem Thema befragt werden soll, und welche Informationen erlangt werden sollen. Dies beinhaltet, welcher Geschäftsprozess untersucht werden soll, und welche Informationen über diesen Prozess gesammelt werden sollen.
    \item \textit{Auswahl der Interviewteilnehmer:} Im nächsten Schritt werden die Interviewteilnehmer ausgewählt. Dabei ist es wichtig, Personen aus verschiedenen Ebenen der Organisation mit unterschiedlichen Perspektiven auf den Prozess einzubeziehen. Dies trägt zur Ermittlung von Zusammenhängen und gegenseitigen Abhängigkeiten zwischen den Prozessen bei.
\end{enumerate}

\paragraph{2. Entwicklung des Interviewleitfadens}\
\begin{enumerate}
\item \textit{Erstellung eines Fragenkatalogs:} Ein Katalog von Fragen wird entwickelt, der auf die Ziele des Interviews abgestimmt ist und alle relevanten Aspekte der Geschäftsprozesse abdeckt. Dabei wird darauf geachtet, dass die Fragen klar, präzise und verständlich formuliert sind.
\item \textit{Planung der Fragenabfolge:} Die Reihenfolge der Fragen wird festgelegt, um einen natürlichen Fluss des Gesprächs zu ermöglichen und sicherzustellen, dass alle Themen in einer sinnvollen Reihenfolge behandelt werden.
\end{enumerate}

\paragraph{3. Durchführung des Interviews}
\begin{enumerate}
\item \textit{Einleitung:} Das Interview beginnt mit einer kurzen Einleitung, bei der die Interviewer sich selbst vorstellen, den Zweck des Interviews erklären und dem Teilnehmer die Möglichkeit geben, sich selbst vorzustellen.
\item \textit{Fragen stellen:} Die im Interviewleitfaden entwickelten Fragen werden in der geplanten Reihenfolge gestellt. Dabei wird darauf geachtet, dass sich die Interviewer an die vorgegebenen Fragen halten, um die Struktur des Interviews beizubehalten und die Vergleichbarkeit der Ergebnisse zu gewährleisten.
\item \textit{Erfassung der Antworten:} Die Antworten der Teilnehmer auf die gestellten Fragen werden systematisch und präzise dokumentiert, entweder schriftlich oder mithilfe einer Tonaufnahme.
\item \textit{Abschluss:} Das Interview endet mit einer Zusammenfassung der wichtigsten Punkte und einer Gelegenheit für die Teilnehmer, zusätzliche Fragen oder Anmerkungen zu äußern. Den Teilnehmern wird für ihre Zeit und ihre Beiträge gedankt.
\end{enumerate}

Insgesamt stellen Interviews eine wichtige Methode in der Business Process Discovery dar, um wertvolle Einblicke in die Abläufe und Herausforderungen von Geschäftsprozessen aus der Perspektive der beteiligten Mitarbeiter zu gewinnen. Eine sorgfältige Planung, Durchführung und Analyse der Interviews ist entscheidend, um die Qualität der Ergebnisse zu gewährleisten und die Identifikation von Verbesserungsmöglichkeiten zu unterstützen.