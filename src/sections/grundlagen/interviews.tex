Interviews dienen als Methode zur Entdeckung von Geschäftsprozessen, indem die Beteiligten in die Prozessausführung einbezogen werden. Diese Technik hat ihre Vor- und Nachteile und kann in verschiedenen Phasen der Prozessentdeckung eingesetzt werden.

\textbf{Vorteile}

\begin{itemize}
    \item \textbf{Gründliche Informationen}: Interviews erleichtern die Sammlung detaillierter Informationen über Geschäftsprozesse, einschließlich der Perspektiven und Erfahrungen der Stakeholder~\cite{Dumas2013}.
    \item \textbf{Flexibilität}: Interviews können auf die Bedürfnisse und Perspektiven der Stakeholder zugeschnitten werden, so dass die Forscher die Fragen auf der Grundlage der Antworten der Teilnehmer anpassen können~\cite{Seidman2006}.
    \item \textbf{Validierung}: Interviews können dazu beitragen, Erkenntnisse aus anderen Datenquellen zu validieren und zu triangulieren, um ein umfassendes Verständnis der Prozesse zu gewährleisten~\cite{Denzin1978}.
    \item \textbf{Implizites Wissen}: Forscher können durch Interviews verstecktes oder implizites Prozesswissen aufdecken, einschließlich nicht dokumentierter Schritte oder informeller Praktiken~\cite{Dumas2013}.
\end{itemize}

\textbf{Nachteile}

\begin{itemize}
    \item \textbf{Subjektivität}: Interviews sind anfällig für Subjektivität und Voreingenommenheit, da die Teilnehmer unterschiedliche Perspektiven oder Interpretationen von Prozessen haben können~\cite{Seidman2006}.
    \item \textbf{Zeitaufwendig}: Die Durchführung und Auswertung von Interviews erfordert einen erheblichen Zeitaufwand für die Planung, Datentranskription und Analyse~\cite{Kvale2009}.
    \item \textbf{Geschränkte Verallgemeinerbarkeit}: Die Interviews werden mit kleineren Stichproben von Interessengruppen durchgeführt, was die Verallgemeinerbarkeit der Ergebnisse einschränken kann~\cite[P. 253]{Creswell2014}
    \item \textbf{Gedächtnis der Teilnehmer}: Interviews beruhen auf den Erinnerungsfähigkeiten der Teilnehmer, die durch Faktoren wie Zeit oder kognitive Verzerrungen beeinflusst werden können, was die Genauigkeit der bereitgestellten Informationen beeinträchtigt~\cite{Kvale2009}.
\end{itemize}

Interviews können in verschiedenen Phasen der Prozessentdeckung eingesetzt werden:

\begin{itemize}
    \item \textbf{Voruntersuchung}: Interviews können erste Informationen über Prozesse liefern, Beteiligte identifizieren und den Kontext für weitere Untersuchungen herstellen~\cite{Dumas2013}.
    \item \textbf{Verfeinerung von Prozessmodellen}: Nach der Erstellung von Prozessmodellen mit Hilfe anderer Methoden können Interviews dazu beitragen, diese Modelle zu validieren und zu verfeinern, indem Erkenntnisse der Stakeholder einbezogen werden und verstecktes oder implizites Prozesswissen aufgedeckt wird~\cite{Dumas2013}.
    \item \textbf{Evaluierung der Prozessleistung}: Mit Hilfe von Interviews lassen sich Rückmeldungen der Stakeholder zur Prozessleistung einholen, verbesserungswürdige Bereiche aufzeigen und potenzielle Lösungen untersuchen~\cite{Dumas2013}.
\end{itemize}

Zusammenfassend lässt sich sagen, dass Interviews eine wichtige Methode sind, um Geschäftsprozesse zu erforschen und zu verstehen, indem man sich mit Interessengruppen auseinandersetzt, die direkt an der Prozessausführung beteiligt sind. 

\subsubsection{Stakeholder-Kategorisierung}

Kategorisiere Stakeholder basierend auf ihren Rollen und Verantwortlichkeiten im Prozess.
Dies ermöglicht es dem Analysten, Fragen auf die spezifischen Bedürfnisse und Perspektiven jeder Gruppe zuzuschneiden, um eine umfassende und relevante Datenerhebung sicherzustellen\cite{Dumas2013}.

\subsubsection{Analysesziele definieren}

Skizziere die Ziele und Objektive der Analyse und identifiziere die spezifischen Aspekte des Prozesses, die untersucht werden sollen, sowie die Art der Informationen, die von jeder Stakeholder-Gruppe gesammelt werden sollen~\cite{Seidman2006}.

\subsubsection{Offene Fragen entwickeln}

Entwerfe offene Fragen, die detaillierte Antworten fördern und den Teilnehmenden ermöglichen, ihre Erfahrungen, Erkenntnisse und Meinungen zu teilen~\cite{Kvale2009}. 
Offene Fragen helfen, suggestive oder voreingenommene Fragen zu vermeiden, die die Antworten der Teilnehmenden beeinflussen könnten.

\subsubsection{Fokus auf den Prozess}

Stelle sicher, dass die Interviewfragen relevant für die Analysesziele sind und sich auf den Prozess selbst konzentrieren, einschließlich Input, Aktivitäten und Output sowie Faktoren, die den Prozess beeinflussen, wie Ressourcen, Werkzeuge, Technologien und Abhängigkeiten von anderen Prozessen~\cite{Dumas2013}.

\subsubsection{Nachfragen vorbereiten}

Entwerfe Nachfragen, um bestimmte Themen basierend auf der ursprünglichen Antwort des Teilnehmenden vertiefter zu untersuchen. Nachfragen helfen, Unklarheiten oder Inkonsistenzen in der ursprünglichen Antwort des Teilnehmenden zu klären und gewährleisten eine genaue und verlässliche Datenerhebung~\cite{Kvale2009}.