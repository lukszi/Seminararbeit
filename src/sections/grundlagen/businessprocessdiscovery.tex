Die Identifikation von Geschäftsprozessen ist der erste Schritt zur Erstellung eines Modells auf dessen Basis weiter gearbeitet werden kann.
Im Laufe der Jahre wurden verschiedene Methoden und Modelle zur Prozessidentifikation entwickelt, im Folgenden werden einige der verbreitetsten Ansätze vorgestellt:

\subsubsection{Manuelle Techniken}\label{subsubsec:discovery-manuelle-techniken}

\begin{itemize}
   \item \textbf{Interviews}: Befragung von Stakeholdern zur Sammlung von Informationen über Prozesse~\cite{Kvale2009}, die reichhaltige qualitative Daten und Einblicke von Teilnehmern liefern und es dem Analyst ermöglichen, Fragen auf die spezifischen Bedürfnisse und Perspektiven jeder Interessengruppe zuzuschneiden. 
   \item \textbf{Workshops}: Ähnlich wie bei Interviews geht es bei Workshops darum, Informationen über Prozesse in einem Gruppenrahmen zu sammeln, um die Zusammenarbeit und Diskussion zwischen den Teilnehmern zu fördern und ein gemeinsames Verständnis des Prozesses zu entwickeln.
   \item \textbf{Observation}: Bei der direkten Beobachtung beobachtet und dokumentiert der Analyst die Aktivitäten des Prozesses in Echtzeit. Dadurch erhält er einen Bericht aus erster Hand über die Ausführung des Prozesses und kann potenzielle Lücken zwischen den dokumentierten Verfahren und der tatsächlichen Praxis aufdecken.
   \item \textbf{Dokumentenanalyse}: Untersuchung von Dokumenten, die sich auf den Prozess beziehen, wie Handbücher, Richtlinien und Formulare, um Informationen zu sammeln~\cite{Seidman2006}, die Einblicke in das beabsichtigte Prozessdesign und die formalen Regeln für die Prozessausführung bieten.
 \end{itemize}

\subsubsection{Automatisierte Techniken}

\begin{itemize}
   \item \textbf{Process Mining}: Analyse von Ereignisprotokollen, die von Informationssystemen generiert werden, um Geschäftsprozesse zu entdecken und zu modellieren (van der Aalst, 2011), um eine objektive und genaue Darstellung des Prozesses auf der Grundlage der tatsächlichen Ausführungsdaten zu erhalten und um Konformitätsprüfungen und Leistungsanalysen zu ermöglichen.
   \item \textbf{Maschinelles Lernen und künstliche Intelligenz}: Anwendung von Techniken des maschinellen Lernens und der künstlichen Intelligenz zur Analyse unstrukturierter Daten und zur Ableitung von Prozessinformationen (Leopold et al., 2018), wodurch die Entdeckung von Prozessmodellen aus Quellen wie Text, Bildern oder Videos ermöglicht wird und potenziell verstecktes oder implizites Prozesswissen aufgedeckt wird.
\end{itemize}

\textbf{References}
