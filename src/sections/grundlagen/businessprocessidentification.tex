\subsection{Methoden der Business Process Identification}

Die Identifikation von Geschäftsprozessen ist ein entscheidender Schritt für die Analyse, Modellierung und Optimierung von Geschäftsabläufen. Es gibt verschiedene Ansätze, um Geschäftsprozesse in einer Organisation zu identifizieren. In diesem Abschnitt wird eine Übersicht über die gängigsten Methoden gegeben.

\textbf{Interviews und Workshops:} Eine der häufigsten Methoden, um Geschäftsprozesse zu identifizieren, ist die Durchführung von Interviews und Workshops mit den beteiligten Stakeholdern. Durch diese direkte Interaktion können Informationen über die aktuellen Geschäftsabläufe, Herausforderungen und Verbesserungspotenziale gesammelt werden~\cite[5.2.2, 5.2.3]{Dumas2013}.

\textbf{Dokumentenanalyse:} Eine weitere Methode zur Identifikation von Geschäftsprozessen ist die Analyse von vorhandenen Dokumenten, wie z.B. Organigrammen, Verfahrensanweisungen, Arbeitsanweisungen und anderen relevanten Unterlagen. Diese Analyse ermöglicht es, einen Einblick in die Struktur und den Ablauf von Geschäftsprozessen zu gewinnen~\cite[5.2.1]{Dumas2013}.

\textbf{Observation:} Die Beobachtung von Mitarbeitern bei der Ausführung ihrer Aufgaben kann ebenfalls helfen, Geschäftsprozesse zu identifizieren. Diese Methode ermöglicht es, die tatsächlichen Abläufe und Interaktionen zwischen verschiedenen Abteilungen und Rollen in der Organisation zu verstehen.~\cite[5.2.1]{Dumas2013}

\textbf{Process Mining:} Process Mining ist eine datengetriebene Methode zur Identifikation von Geschäftsprozessen, bei der Ereignisprotokolle aus Informationssystemen analysiert werden. Mithilfe dieser Methode kann eine besonders gut skalierbare und wiederholbare Identifikation in großen Unternehmen und Abteilungen vorgenommen werden, wo andere Techniken oft zu Zeitintensiv wären.~\cite{Aalst2012}

In der Praxis wird häufig eine Kombination dieser Methoden verwendet, um ein umfassendes Verständnis der Geschäftsprozesse einer Organisation zu gewinnen und eine solide Grundlage für die weitere Analyse und Verbesserung zu schaffen.